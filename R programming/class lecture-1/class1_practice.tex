% Options for packages loaded elsewhere
\PassOptionsToPackage{unicode}{hyperref}
\PassOptionsToPackage{hyphens}{url}
%
\documentclass[
]{article}
\usepackage{amsmath,amssymb}
\usepackage{iftex}
\ifPDFTeX
  \usepackage[T1]{fontenc}
  \usepackage[utf8]{inputenc}
  \usepackage{textcomp} % provide euro and other symbols
\else % if luatex or xetex
  \usepackage{unicode-math} % this also loads fontspec
  \defaultfontfeatures{Scale=MatchLowercase}
  \defaultfontfeatures[\rmfamily]{Ligatures=TeX,Scale=1}
\fi
\usepackage{lmodern}
\ifPDFTeX\else
  % xetex/luatex font selection
\fi
% Use upquote if available, for straight quotes in verbatim environments
\IfFileExists{upquote.sty}{\usepackage{upquote}}{}
\IfFileExists{microtype.sty}{% use microtype if available
  \usepackage[]{microtype}
  \UseMicrotypeSet[protrusion]{basicmath} % disable protrusion for tt fonts
}{}
\makeatletter
\@ifundefined{KOMAClassName}{% if non-KOMA class
  \IfFileExists{parskip.sty}{%
    \usepackage{parskip}
  }{% else
    \setlength{\parindent}{0pt}
    \setlength{\parskip}{6pt plus 2pt minus 1pt}}
}{% if KOMA class
  \KOMAoptions{parskip=half}}
\makeatother
\usepackage{xcolor}
\usepackage[margin=1in]{geometry}
\usepackage{color}
\usepackage{fancyvrb}
\newcommand{\VerbBar}{|}
\newcommand{\VERB}{\Verb[commandchars=\\\{\}]}
\DefineVerbatimEnvironment{Highlighting}{Verbatim}{commandchars=\\\{\}}
% Add ',fontsize=\small' for more characters per line
\usepackage{framed}
\definecolor{shadecolor}{RGB}{248,248,248}
\newenvironment{Shaded}{\begin{snugshade}}{\end{snugshade}}
\newcommand{\AlertTok}[1]{\textcolor[rgb]{0.94,0.16,0.16}{#1}}
\newcommand{\AnnotationTok}[1]{\textcolor[rgb]{0.56,0.35,0.01}{\textbf{\textit{#1}}}}
\newcommand{\AttributeTok}[1]{\textcolor[rgb]{0.13,0.29,0.53}{#1}}
\newcommand{\BaseNTok}[1]{\textcolor[rgb]{0.00,0.00,0.81}{#1}}
\newcommand{\BuiltInTok}[1]{#1}
\newcommand{\CharTok}[1]{\textcolor[rgb]{0.31,0.60,0.02}{#1}}
\newcommand{\CommentTok}[1]{\textcolor[rgb]{0.56,0.35,0.01}{\textit{#1}}}
\newcommand{\CommentVarTok}[1]{\textcolor[rgb]{0.56,0.35,0.01}{\textbf{\textit{#1}}}}
\newcommand{\ConstantTok}[1]{\textcolor[rgb]{0.56,0.35,0.01}{#1}}
\newcommand{\ControlFlowTok}[1]{\textcolor[rgb]{0.13,0.29,0.53}{\textbf{#1}}}
\newcommand{\DataTypeTok}[1]{\textcolor[rgb]{0.13,0.29,0.53}{#1}}
\newcommand{\DecValTok}[1]{\textcolor[rgb]{0.00,0.00,0.81}{#1}}
\newcommand{\DocumentationTok}[1]{\textcolor[rgb]{0.56,0.35,0.01}{\textbf{\textit{#1}}}}
\newcommand{\ErrorTok}[1]{\textcolor[rgb]{0.64,0.00,0.00}{\textbf{#1}}}
\newcommand{\ExtensionTok}[1]{#1}
\newcommand{\FloatTok}[1]{\textcolor[rgb]{0.00,0.00,0.81}{#1}}
\newcommand{\FunctionTok}[1]{\textcolor[rgb]{0.13,0.29,0.53}{\textbf{#1}}}
\newcommand{\ImportTok}[1]{#1}
\newcommand{\InformationTok}[1]{\textcolor[rgb]{0.56,0.35,0.01}{\textbf{\textit{#1}}}}
\newcommand{\KeywordTok}[1]{\textcolor[rgb]{0.13,0.29,0.53}{\textbf{#1}}}
\newcommand{\NormalTok}[1]{#1}
\newcommand{\OperatorTok}[1]{\textcolor[rgb]{0.81,0.36,0.00}{\textbf{#1}}}
\newcommand{\OtherTok}[1]{\textcolor[rgb]{0.56,0.35,0.01}{#1}}
\newcommand{\PreprocessorTok}[1]{\textcolor[rgb]{0.56,0.35,0.01}{\textit{#1}}}
\newcommand{\RegionMarkerTok}[1]{#1}
\newcommand{\SpecialCharTok}[1]{\textcolor[rgb]{0.81,0.36,0.00}{\textbf{#1}}}
\newcommand{\SpecialStringTok}[1]{\textcolor[rgb]{0.31,0.60,0.02}{#1}}
\newcommand{\StringTok}[1]{\textcolor[rgb]{0.31,0.60,0.02}{#1}}
\newcommand{\VariableTok}[1]{\textcolor[rgb]{0.00,0.00,0.00}{#1}}
\newcommand{\VerbatimStringTok}[1]{\textcolor[rgb]{0.31,0.60,0.02}{#1}}
\newcommand{\WarningTok}[1]{\textcolor[rgb]{0.56,0.35,0.01}{\textbf{\textit{#1}}}}
\usepackage{graphicx}
\makeatletter
\def\maxwidth{\ifdim\Gin@nat@width>\linewidth\linewidth\else\Gin@nat@width\fi}
\def\maxheight{\ifdim\Gin@nat@height>\textheight\textheight\else\Gin@nat@height\fi}
\makeatother
% Scale images if necessary, so that they will not overflow the page
% margins by default, and it is still possible to overwrite the defaults
% using explicit options in \includegraphics[width, height, ...]{}
\setkeys{Gin}{width=\maxwidth,height=\maxheight,keepaspectratio}
% Set default figure placement to htbp
\makeatletter
\def\fps@figure{htbp}
\makeatother
\setlength{\emergencystretch}{3em} % prevent overfull lines
\providecommand{\tightlist}{%
  \setlength{\itemsep}{0pt}\setlength{\parskip}{0pt}}
\setcounter{secnumdepth}{-\maxdimen} % remove section numbering
\ifLuaTeX
  \usepackage{selnolig}  % disable illegal ligatures
\fi
\IfFileExists{bookmark.sty}{\usepackage{bookmark}}{\usepackage{hyperref}}
\IfFileExists{xurl.sty}{\usepackage{xurl}}{} % add URL line breaks if available
\urlstyle{same}
\hypersetup{
  pdftitle={R Notebook-CLASS-1},
  hidelinks,
  pdfcreator={LaTeX via pandoc}}

\title{R Notebook-CLASS-1}
\author{}
\date{\vspace{-2.5em}}

\begin{document}
\maketitle

Try executing this chunk by clicking the \emph{Run} button within the
chunk or by placing your cursor inside it and pressing
\emph{Ctrl+Shift+Enter}.

\begin{verbatim}
31*78
\end{verbatim}

Add a new chunk by clicking the \emph{Insert Chunk} button on the
toolbar or by pressing \emph{Ctrl+Alt+I}.

When you save the notebook, an HTML file containing the code and output
will be saved alongside it (click the \emph{Preview} button or press
\emph{Ctrl+Shift+K} to preview the HTML file).

The preview shows you a rendered HTML copy of the contents of the
editor. Consequently, unlike \emph{Knit}, \emph{Preview} does not run
any R code chunks. Instead, the output of the chunk when it was last run
in the editor is displayed.

\hypertarget{ex-1}{%
\section{Ex-1}\label{ex-1}}

\begin{Shaded}
\begin{Highlighting}[]
\NormalTok{a }\OtherTok{\textless{}{-}} \DecValTok{5}\SpecialCharTok{*}\DecValTok{3}
\NormalTok{a}
\end{Highlighting}
\end{Shaded}

\begin{verbatim}
## [1] 15
\end{verbatim}

\begin{Shaded}
\begin{Highlighting}[]
\FunctionTok{help}\NormalTok{(log)}
\end{Highlighting}
\end{Shaded}

\begin{verbatim}
## starting httpd help server ... done
\end{verbatim}

\begin{Shaded}
\begin{Highlighting}[]
\NormalTok{a}\OtherTok{\textless{}{-}}\DecValTok{1}
\NormalTok{b}\OtherTok{\textless{}{-}}\NormalTok{ (}\SpecialCharTok{{-}}\DecValTok{6}\NormalTok{)}
\NormalTok{c}\OtherTok{\textless{}{-}}\DecValTok{1}
\NormalTok{s1}\OtherTok{\textless{}{-}}\NormalTok{ (}\SpecialCharTok{{-}}\NormalTok{b}\SpecialCharTok{+}\FunctionTok{sqrt}\NormalTok{(b}\SpecialCharTok{\^{}}\DecValTok{2} \SpecialCharTok{{-}} \DecValTok{4}\SpecialCharTok{*}\NormalTok{a}\SpecialCharTok{*}\NormalTok{c))}\SpecialCharTok{/}\NormalTok{(}\DecValTok{2}\SpecialCharTok{*}\NormalTok{a)}
\NormalTok{s2}\OtherTok{\textless{}{-}}\NormalTok{ (}\SpecialCharTok{{-}}\NormalTok{b}\SpecialCharTok{{-}}\FunctionTok{sqrt}\NormalTok{(b}\SpecialCharTok{\^{}}\DecValTok{2} \SpecialCharTok{{-}} \DecValTok{4}\SpecialCharTok{*}\NormalTok{a}\SpecialCharTok{*}\NormalTok{c))}\SpecialCharTok{/}\NormalTok{(}\DecValTok{2}\SpecialCharTok{*}\NormalTok{a)}
\FunctionTok{print}\NormalTok{(s1)}
\end{Highlighting}
\end{Shaded}

\begin{verbatim}
## [1] 5.828427
\end{verbatim}

\begin{Shaded}
\begin{Highlighting}[]
\FunctionTok{print}\NormalTok{(s2)}
\end{Highlighting}
\end{Shaded}

\begin{verbatim}
## [1] 0.1715729
\end{verbatim}

\hypertarget{vectros}{%
\section{Vectros}\label{vectros}}

\begin{Shaded}
\begin{Highlighting}[]
\NormalTok{vector1}\OtherTok{\textless{}{-}} \FunctionTok{c}\NormalTok{(}\DecValTok{1}\NormalTok{,}\DecValTok{2}\NormalTok{,}\DecValTok{3}\NormalTok{)}
\NormalTok{vector1}
\end{Highlighting}
\end{Shaded}

\begin{verbatim}
## [1] 1 2 3
\end{verbatim}

\begin{Shaded}
\begin{Highlighting}[]
\FunctionTok{class}\NormalTok{(vector1)}
\end{Highlighting}
\end{Shaded}

\begin{verbatim}
## [1] "numeric"
\end{verbatim}

\begin{Shaded}
\begin{Highlighting}[]
\NormalTok{vect2}\OtherTok{\textless{}{-}} \FunctionTok{c}\NormalTok{(}\StringTok{\textquotesingle{}i\textquotesingle{}}\NormalTok{,}\StringTok{\textquotesingle{}n\textquotesingle{}}\NormalTok{,}\StringTok{\textquotesingle{}d\textquotesingle{}}\NormalTok{,}\StringTok{\textquotesingle{}i\textquotesingle{}}\NormalTok{,}\StringTok{\textquotesingle{}a\textquotesingle{}}\NormalTok{)}
\NormalTok{vect2}
\end{Highlighting}
\end{Shaded}

\begin{verbatim}
## [1] "i" "n" "d" "i" "a"
\end{verbatim}

\begin{Shaded}
\begin{Highlighting}[]
\FunctionTok{class}\NormalTok{(vect2)}
\end{Highlighting}
\end{Shaded}

\begin{verbatim}
## [1] "character"
\end{verbatim}

\begin{Shaded}
\begin{Highlighting}[]
\NormalTok{vect3}\OtherTok{\textless{}{-}}\FunctionTok{c}\NormalTok{(}\DecValTok{1}\NormalTok{,}\DecValTok{2}\NormalTok{,}\StringTok{\textquotesingle{}i\textquotesingle{}}\NormalTok{,}\StringTok{\textquotesingle{}p\textquotesingle{}}\NormalTok{)}
\NormalTok{vect3}
\end{Highlighting}
\end{Shaded}

\begin{verbatim}
## [1] "1" "2" "i" "p"
\end{verbatim}

\begin{Shaded}
\begin{Highlighting}[]
\FunctionTok{class}\NormalTok{(vect3)}
\end{Highlighting}
\end{Shaded}

\begin{verbatim}
## [1] "character"
\end{verbatim}

\begin{Shaded}
\begin{Highlighting}[]
\NormalTok{vect4}\OtherTok{\textless{}{-}}\FunctionTok{c}\NormalTok{(}\StringTok{"italy"}\OtherTok{=}\DecValTok{10}\NormalTok{,}\StringTok{"canada"}\OtherTok{=}\DecValTok{20}\NormalTok{,}\StringTok{"usa"}\OtherTok{=}\DecValTok{30}\NormalTok{ )}
\NormalTok{vect4}
\end{Highlighting}
\end{Shaded}

\begin{verbatim}
##  italy canada    usa 
##     10     20     30
\end{verbatim}

\begin{Shaded}
\begin{Highlighting}[]
\FunctionTok{class}\NormalTok{(vect4)}
\end{Highlighting}
\end{Shaded}

\begin{verbatim}
## [1] "numeric"
\end{verbatim}

\begin{Shaded}
\begin{Highlighting}[]
\FunctionTok{seq}\NormalTok{(}\DecValTok{1}\NormalTok{,}\DecValTok{10}\NormalTok{)}
\end{Highlighting}
\end{Shaded}

\begin{verbatim}
##  [1]  1  2  3  4  5  6  7  8  9 10
\end{verbatim}

\begin{Shaded}
\begin{Highlighting}[]
\DecValTok{1}\SpecialCharTok{:}\DecValTok{5}
\end{Highlighting}
\end{Shaded}

\begin{verbatim}
## [1] 1 2 3 4 5
\end{verbatim}

\begin{Shaded}
\begin{Highlighting}[]
\FunctionTok{names}\NormalTok{(vect4)}
\end{Highlighting}
\end{Shaded}

\begin{verbatim}
## [1] "italy"  "canada" "usa"
\end{verbatim}

\begin{Shaded}
\begin{Highlighting}[]
\NormalTok{vect4[}\StringTok{"italy"}\NormalTok{]}
\end{Highlighting}
\end{Shaded}

\begin{verbatim}
## italy 
##    10
\end{verbatim}

\begin{Shaded}
\begin{Highlighting}[]
\NormalTok{vect4[}\DecValTok{2}\NormalTok{]}
\end{Highlighting}
\end{Shaded}

\begin{verbatim}
## canada 
##     20
\end{verbatim}

\begin{Shaded}
\begin{Highlighting}[]
\NormalTok{vect4[}\DecValTok{1}\SpecialCharTok{:}\DecValTok{3}\NormalTok{]}
\end{Highlighting}
\end{Shaded}

\begin{verbatim}
##  italy canada    usa 
##     10     20     30
\end{verbatim}

\begin{Shaded}
\begin{Highlighting}[]
\NormalTok{vect4[}\FunctionTok{c}\NormalTok{(}\DecValTok{1}\NormalTok{,}\DecValTok{3}\NormalTok{)]}
\end{Highlighting}
\end{Shaded}

\begin{verbatim}
## italy   usa 
##    10    30
\end{verbatim}

\begin{Shaded}
\begin{Highlighting}[]
\NormalTok{a}\OtherTok{\textless{}{-}} \FunctionTok{c}\NormalTok{(}\DecValTok{1}\NormalTok{,}\StringTok{"canada"}\NormalTok{,}\DecValTok{2}\NormalTok{)}
\FunctionTok{class}\NormalTok{(a)}
\end{Highlighting}
\end{Shaded}

\begin{verbatim}
## [1] "character"
\end{verbatim}

\begin{Shaded}
\begin{Highlighting}[]
\NormalTok{vect3}\OtherTok{\textless{}{-}}\FunctionTok{c}\NormalTok{(}\DecValTok{1}\NormalTok{,}\DecValTok{2}\NormalTok{,}\StringTok{\textquotesingle{}i\textquotesingle{}}\NormalTok{,}\StringTok{\textquotesingle{}v\textquotesingle{}}\NormalTok{)}
\FunctionTok{class}\NormalTok{(vect3)}
\end{Highlighting}
\end{Shaded}

\begin{verbatim}
## [1] "character"
\end{verbatim}

\begin{Shaded}
\begin{Highlighting}[]
\NormalTok{x}\OtherTok{\textless{}{-}} \DecValTok{1}\SpecialCharTok{:}\DecValTok{10}
\NormalTok{x}
\end{Highlighting}
\end{Shaded}

\begin{verbatim}
##  [1]  1  2  3  4  5  6  7  8  9 10
\end{verbatim}

\begin{Shaded}
\begin{Highlighting}[]
\NormalTok{y}\OtherTok{\textless{}{-}} \FunctionTok{as.character}\NormalTok{(x)}
\NormalTok{y}
\end{Highlighting}
\end{Shaded}

\begin{verbatim}
##  [1] "1"  "2"  "3"  "4"  "5"  "6"  "7"  "8"  "9"  "10"
\end{verbatim}

\begin{Shaded}
\begin{Highlighting}[]
\FunctionTok{as.numeric}\NormalTok{(y)}
\end{Highlighting}
\end{Shaded}

\begin{verbatim}
##  [1]  1  2  3  4  5  6  7  8  9 10
\end{verbatim}

\hypertarget{matrix}{%
\section{matrix}\label{matrix}}

\begin{Shaded}
\begin{Highlighting}[]
\NormalTok{mat}\OtherTok{\textless{}{-}} \FunctionTok{matrix}\NormalTok{(}\DecValTok{1}\SpecialCharTok{:}\DecValTok{12}\NormalTok{,}\AttributeTok{nrow =} \DecValTok{4}\NormalTok{,}\AttributeTok{ncol =} \DecValTok{3}\NormalTok{)}
\NormalTok{mat}
\end{Highlighting}
\end{Shaded}

\begin{verbatim}
##      [,1] [,2] [,3]
## [1,]    1    5    9
## [2,]    2    6   10
## [3,]    3    7   11
## [4,]    4    8   12
\end{verbatim}

\begin{Shaded}
\begin{Highlighting}[]
\NormalTok{mat[}\DecValTok{2}\NormalTok{,}\DecValTok{2}\NormalTok{]}
\end{Highlighting}
\end{Shaded}

\begin{verbatim}
## [1] 6
\end{verbatim}

\begin{Shaded}
\begin{Highlighting}[]
\NormalTok{mat[}\DecValTok{2}\NormalTok{,]}
\end{Highlighting}
\end{Shaded}

\begin{verbatim}
## [1]  2  6 10
\end{verbatim}

\begin{Shaded}
\begin{Highlighting}[]
\NormalTok{mat[,}\DecValTok{3}\NormalTok{]}
\end{Highlighting}
\end{Shaded}

\begin{verbatim}
## [1]  9 10 11 12
\end{verbatim}

\begin{Shaded}
\begin{Highlighting}[]
\NormalTok{mat[,}\DecValTok{2}\SpecialCharTok{:}\DecValTok{3}\NormalTok{]}
\end{Highlighting}
\end{Shaded}

\begin{verbatim}
##      [,1] [,2]
## [1,]    5    9
## [2,]    6   10
## [3,]    7   11
## [4,]    8   12
\end{verbatim}

\begin{Shaded}
\begin{Highlighting}[]
\NormalTok{mat[}\DecValTok{1}\NormalTok{,}\DecValTok{2}\SpecialCharTok{:}\DecValTok{3}\NormalTok{]}
\end{Highlighting}
\end{Shaded}

\begin{verbatim}
## [1] 5 9
\end{verbatim}

\begin{Shaded}
\begin{Highlighting}[]
\FunctionTok{as.data.frame}\NormalTok{(mat)}
\end{Highlighting}
\end{Shaded}

\begin{verbatim}
##   V1 V2 V3
## 1  1  5  9
## 2  2  6 10
## 3  3  7 11
## 4  4  8 12
\end{verbatim}

\hypertarget{list}{%
\section{list}\label{list}}

\begin{Shaded}
\begin{Highlighting}[]
\NormalTok{list1}\OtherTok{=}\FunctionTok{list}\NormalTok{(}\DecValTok{1}\NormalTok{,}\DecValTok{2}\NormalTok{,}\DecValTok{3}\NormalTok{)}
\NormalTok{list1}
\end{Highlighting}
\end{Shaded}

\begin{verbatim}
## [[1]]
## [1] 1
## 
## [[2]]
## [1] 2
## 
## [[3]]
## [1] 3
\end{verbatim}

\begin{Shaded}
\begin{Highlighting}[]
\NormalTok{mylist1}\OtherTok{=}\FunctionTok{list}\NormalTok{(}\AttributeTok{name=}\StringTok{"prabal"}\NormalTok{,}\AttributeTok{degree=}\StringTok{"msc"}\NormalTok{,}\AttributeTok{course=}\StringTok{"dsai"}\NormalTok{)}
\NormalTok{mylist1}
\end{Highlighting}
\end{Shaded}

\begin{verbatim}
## $name
## [1] "prabal"
## 
## $degree
## [1] "msc"
## 
## $course
## [1] "dsai"
\end{verbatim}

\begin{Shaded}
\begin{Highlighting}[]
\NormalTok{out}\OtherTok{=}\FunctionTok{t.test}\NormalTok{(}\DecValTok{1}\SpecialCharTok{:}\DecValTok{10}\NormalTok{,}\FunctionTok{c}\NormalTok{(}\DecValTok{7}\SpecialCharTok{:}\DecValTok{20}\NormalTok{))}
\NormalTok{out}
\end{Highlighting}
\end{Shaded}

\begin{verbatim}
## 
##  Welch Two Sample t-test
## 
## data:  1:10 and c(7:20)
## t = -5.4349, df = 21.982, p-value = 1.855e-05
## alternative hypothesis: true difference in means is not equal to 0
## 95 percent confidence interval:
##  -11.052802  -4.947198
## sample estimates:
## mean of x mean of y 
##       5.5      13.5
\end{verbatim}

\begin{Shaded}
\begin{Highlighting}[]
\NormalTok{a}\OtherTok{\textless{}{-}}\FunctionTok{c}\NormalTok{(}\DecValTok{1}\NormalTok{,}\DecValTok{2}\NormalTok{,}\DecValTok{3}\NormalTok{,}\DecValTok{4}\NormalTok{)}
\NormalTok{b}\OtherTok{\textless{}{-}}\FunctionTok{c}\NormalTok{(}\StringTok{"d"}\NormalTok{,}\StringTok{"e"}\NormalTok{,}\StringTok{"f"}\NormalTok{,}\StringTok{"g"}\NormalTok{)}
\NormalTok{c}\OtherTok{\textless{}{-}}\FunctionTok{c}\NormalTok{(}\StringTok{"hi"}\NormalTok{,}\StringTok{"bp"}\NormalTok{,}\StringTok{"np"}\NormalTok{,}\StringTok{"gh"}\NormalTok{)}
\NormalTok{df}\OtherTok{\textless{}{-}}\FunctionTok{data.frame}\NormalTok{(a,b,c)}
\NormalTok{df}
\end{Highlighting}
\end{Shaded}

\begin{verbatim}
##   a b  c
## 1 1 d hi
## 2 2 e bp
## 3 3 f np
## 4 4 g gh
\end{verbatim}

\begin{Shaded}
\begin{Highlighting}[]
\NormalTok{df[}\DecValTok{1}\SpecialCharTok{:}\DecValTok{3}\NormalTok{,}\DecValTok{1}\SpecialCharTok{:}\DecValTok{2}\NormalTok{]}
\end{Highlighting}
\end{Shaded}

\begin{verbatim}
##   a b
## 1 1 d
## 2 2 e
## 3 3 f
\end{verbatim}

\begin{Shaded}
\begin{Highlighting}[]
\NormalTok{df[}\DecValTok{1}\NormalTok{,}\DecValTok{2}\NormalTok{]}
\end{Highlighting}
\end{Shaded}

\begin{verbatim}
## [1] "d"
\end{verbatim}

\begin{Shaded}
\begin{Highlighting}[]
\NormalTok{df[}\DecValTok{1}\NormalTok{,]}
\end{Highlighting}
\end{Shaded}

\begin{verbatim}
##   a b  c
## 1 1 d hi
\end{verbatim}

\begin{Shaded}
\begin{Highlighting}[]
\NormalTok{df[,}\DecValTok{3}\NormalTok{]}
\end{Highlighting}
\end{Shaded}

\begin{verbatim}
## [1] "hi" "bp" "np" "gh"
\end{verbatim}

\begin{Shaded}
\begin{Highlighting}[]
\NormalTok{new\_c1}\OtherTok{\textless{}{-}} \FunctionTok{c}\NormalTok{(}\StringTok{"kol"}\NormalTok{,}\StringTok{"hol"}\NormalTok{,}\StringTok{"jol"}\NormalTok{,}\StringTok{"pol"}\NormalTok{)}
\NormalTok{df}\SpecialCharTok{$}\NormalTok{city}\OtherTok{\textless{}{-}}\NormalTok{ new\_c1}
\NormalTok{df}
\end{Highlighting}
\end{Shaded}

\begin{verbatim}
##   a b  c city
## 1 1 d hi  kol
## 2 2 e bp  hol
## 3 3 f np  jol
## 4 4 g gh  pol
\end{verbatim}

\begin{Shaded}
\begin{Highlighting}[]
\FunctionTok{subset}\NormalTok{(df,a}\SpecialCharTok{\textgreater{}}\DecValTok{2}\NormalTok{)}
\end{Highlighting}
\end{Shaded}

\begin{verbatim}
##   a b  c city
## 3 3 f np  jol
## 4 4 g gh  pol
\end{verbatim}

\hypertarget{import-library}{%
\section{Import Library}\label{import-library}}

\begin{Shaded}
\begin{Highlighting}[]
\FunctionTok{library}\NormalTok{(dslabs)}
\FunctionTok{data}\NormalTok{(murders)}
\FunctionTok{class}\NormalTok{(murders)}
\end{Highlighting}
\end{Shaded}

\begin{verbatim}
## [1] "data.frame"
\end{verbatim}

\begin{Shaded}
\begin{Highlighting}[]
\FunctionTok{str}\NormalTok{(murders)}
\end{Highlighting}
\end{Shaded}

\begin{verbatim}
## 'data.frame':    51 obs. of  5 variables:
##  $ state     : chr  "Alabama" "Alaska" "Arizona" "Arkansas" ...
##  $ abb       : chr  "AL" "AK" "AZ" "AR" ...
##  $ region    : Factor w/ 4 levels "Northeast","South",..: 2 4 4 2 4 4 1 2 2 2 ...
##  $ population: num  4779736 710231 6392017 2915918 37253956 ...
##  $ total     : num  135 19 232 93 1257 ...
\end{verbatim}

\begin{Shaded}
\begin{Highlighting}[]
\FunctionTok{head}\NormalTok{(murders)}
\end{Highlighting}
\end{Shaded}

\begin{verbatim}
##        state abb region population total
## 1    Alabama  AL  South    4779736   135
## 2     Alaska  AK   West     710231    19
## 3    Arizona  AZ   West    6392017   232
## 4   Arkansas  AR  South    2915918    93
## 5 California  CA   West   37253956  1257
## 6   Colorado  CO   West    5029196    65
\end{verbatim}

\begin{Shaded}
\begin{Highlighting}[]
\FunctionTok{names}\NormalTok{(murders)}
\end{Highlighting}
\end{Shaded}

\begin{verbatim}
## [1] "state"      "abb"        "region"     "population" "total"
\end{verbatim}

\begin{Shaded}
\begin{Highlighting}[]
\NormalTok{murders}\SpecialCharTok{$}\NormalTok{population}
\end{Highlighting}
\end{Shaded}

\begin{verbatim}
##  [1]  4779736   710231  6392017  2915918 37253956  5029196  3574097   897934
##  [9]   601723 19687653  9920000  1360301  1567582 12830632  6483802  3046355
## [17]  2853118  4339367  4533372  1328361  5773552  6547629  9883640  5303925
## [25]  2967297  5988927   989415  1826341  2700551  1316470  8791894  2059179
## [33] 19378102  9535483   672591 11536504  3751351  3831074 12702379  1052567
## [41]  4625364   814180  6346105 25145561  2763885   625741  8001024  6724540
## [49]  1852994  5686986   563626
\end{verbatim}

\begin{Shaded}
\begin{Highlighting}[]
\FunctionTok{plot}\NormalTok{(}\DecValTok{1}\SpecialCharTok{:}\DecValTok{10}\NormalTok{,(}\DecValTok{1}\SpecialCharTok{:}\DecValTok{10}\NormalTok{)}\SpecialCharTok{\^{}}\DecValTok{4}\NormalTok{)}
\end{Highlighting}
\end{Shaded}

\includegraphics{class1_practice_files/figure-latex/unnamed-chunk-17-1.pdf}

\begin{Shaded}
\begin{Highlighting}[]
\FunctionTok{plot}\NormalTok{(}\DecValTok{1}\SpecialCharTok{:}\DecValTok{10}\NormalTok{,(}\DecValTok{1}\SpecialCharTok{:}\DecValTok{10}\NormalTok{)}\SpecialCharTok{\^{}}\DecValTok{4}\NormalTok{,}\AttributeTok{xlab =} \StringTok{\textquotesingle{}val1\textquotesingle{}}\NormalTok{,}\AttributeTok{ylab =} \StringTok{\textquotesingle{}val2\textquotesingle{}}\NormalTok{,}\AttributeTok{pch=}\DecValTok{19}\NormalTok{,}\AttributeTok{type =} \StringTok{\textquotesingle{}b\textquotesingle{}}\NormalTok{,}\AttributeTok{col=}\StringTok{\textquotesingle{}red\textquotesingle{}}\NormalTok{,}\AttributeTok{main =} \StringTok{"Plot1"}\NormalTok{,}\AttributeTok{cex=}\DecValTok{3}\NormalTok{,}\AttributeTok{lty=}\DecValTok{3}\NormalTok{)}
\end{Highlighting}
\end{Shaded}

\includegraphics{class1_practice_files/figure-latex/unnamed-chunk-18-1.pdf}

\begin{Shaded}
\begin{Highlighting}[]
\FunctionTok{library}\NormalTok{(RColorBrewer)}
\FunctionTok{barplot}\NormalTok{(}\DecValTok{1}\SpecialCharTok{:}\DecValTok{4}\NormalTok{,}\AttributeTok{col=}\FunctionTok{brewer.pal}\NormalTok{(}\DecValTok{4}\NormalTok{,}\StringTok{"Set1"}\NormalTok{))}
\end{Highlighting}
\end{Shaded}

\includegraphics{class1_practice_files/figure-latex/unnamed-chunk-19-1.pdf}

\begin{Shaded}
\begin{Highlighting}[]
\FunctionTok{barplot}\NormalTok{(}\DecValTok{1}\SpecialCharTok{:}\DecValTok{4}\NormalTok{,}\AttributeTok{col=}\FunctionTok{c}\NormalTok{(}\StringTok{"red"}\NormalTok{,}\StringTok{"green"}\NormalTok{))}
\end{Highlighting}
\end{Shaded}

\includegraphics{class1_practice_files/figure-latex/unnamed-chunk-20-1.pdf}

\hypertarget{scatter-plot}{%
\section{Scatter plot}\label{scatter-plot}}

\begin{Shaded}
\begin{Highlighting}[]
\NormalTok{x}\OtherTok{\textless{}{-}}\NormalTok{ murders}\SpecialCharTok{$}\NormalTok{population}\SpecialCharTok{/}\DecValTok{10}\SpecialCharTok{\^{}}\DecValTok{6}
\NormalTok{y}\OtherTok{\textless{}{-}}\NormalTok{ murders}\SpecialCharTok{$}\NormalTok{total}
\FunctionTok{plot}\NormalTok{(x,y,}\AttributeTok{xlim =} \FunctionTok{c}\NormalTok{(}\DecValTok{0}\NormalTok{,}\DecValTok{40}\NormalTok{), }\AttributeTok{ylim =}\FunctionTok{c}\NormalTok{(}\DecValTok{0}\NormalTok{,}\DecValTok{1500}\NormalTok{),}\AttributeTok{pch=}\DecValTok{19}\NormalTok{,}\AttributeTok{xlab =} \StringTok{\textquotesingle{}population\textquotesingle{}}\NormalTok{,}\AttributeTok{ylab =} \StringTok{\textquotesingle{}total\textquotesingle{}}\NormalTok{,}\AttributeTok{col=}\StringTok{\textquotesingle{}red\textquotesingle{}}\NormalTok{,}\AttributeTok{main =} \StringTok{"Murders"}\NormalTok{)}
\FunctionTok{legend}\NormalTok{(}\StringTok{"bottomright"}\NormalTok{,}\AttributeTok{legend =} \StringTok{"data points"}\NormalTok{,}\AttributeTok{pch=}\DecValTok{19}\NormalTok{,}\AttributeTok{col =} \StringTok{"red"}\NormalTok{)}
\FunctionTok{abline}\NormalTok{(}\AttributeTok{a=}\DecValTok{0}\NormalTok{,}\AttributeTok{b=}\DecValTok{30}\NormalTok{,}\AttributeTok{lty=}\DecValTok{2}\NormalTok{,}\AttributeTok{lwd=}\FloatTok{2.5}\NormalTok{,}\AttributeTok{col=}\NormalTok{(}\StringTok{"blue"}\NormalTok{))}
\end{Highlighting}
\end{Shaded}

\includegraphics{class1_practice_files/figure-latex/unnamed-chunk-21-1.pdf}

\hypertarget{box-plot}{%
\section{BOX PLOT}\label{box-plot}}

\begin{Shaded}
\begin{Highlighting}[]
\NormalTok{murders}\SpecialCharTok{$}\NormalTok{rate}\OtherTok{\textless{}{-}} \FunctionTok{with}\NormalTok{(murders,total}\SpecialCharTok{/}\NormalTok{population }\SpecialCharTok{*} \DecValTok{100000}\NormalTok{)}
\FunctionTok{head}\NormalTok{(murders)}
\end{Highlighting}
\end{Shaded}

\begin{verbatim}
##        state abb region population total     rate
## 1    Alabama  AL  South    4779736   135 2.824424
## 2     Alaska  AK   West     710231    19 2.675186
## 3    Arizona  AZ   West    6392017   232 3.629527
## 4   Arkansas  AR  South    2915918    93 3.189390
## 5 California  CA   West   37253956  1257 3.374138
## 6   Colorado  CO   West    5029196    65 1.292453
\end{verbatim}

\begin{Shaded}
\begin{Highlighting}[]
\FunctionTok{boxplot}\NormalTok{(rate}\SpecialCharTok{\textasciitilde{}}\NormalTok{region, }\AttributeTok{data =}\NormalTok{ murders)}
\end{Highlighting}
\end{Shaded}

\includegraphics{class1_practice_files/figure-latex/unnamed-chunk-22-1.pdf}

\hypertarget{histogram}{%
\section{Histogram}\label{histogram}}

\begin{Shaded}
\begin{Highlighting}[]
\NormalTok{x}\OtherTok{\textless{}{-}} \FunctionTok{with}\NormalTok{(murders,total}\SpecialCharTok{/}\NormalTok{population }\SpecialCharTok{*} \DecValTok{100000}\NormalTok{)}
\FunctionTok{hist}\NormalTok{(x)}
\end{Highlighting}
\end{Shaded}

\includegraphics{class1_practice_files/figure-latex/unnamed-chunk-23-1.pdf}

\begin{Shaded}
\begin{Highlighting}[]
\NormalTok{murders}\SpecialCharTok{$}\NormalTok{state[}\FunctionTok{which.max}\NormalTok{(x)]}
\end{Highlighting}
\end{Shaded}

\begin{verbatim}
## [1] "District of Columbia"
\end{verbatim}

\begin{Shaded}
\begin{Highlighting}[]
\NormalTok{x}\OtherTok{=} \FunctionTok{runif}\NormalTok{(}\DecValTok{100}\NormalTok{,}\DecValTok{0}\NormalTok{,}\DecValTok{1}\NormalTok{)}
\FunctionTok{par}\NormalTok{(}\AttributeTok{mflow=}\FunctionTok{c}\NormalTok{(}\DecValTok{1}\NormalTok{,}\DecValTok{2}\NormalTok{))}
\end{Highlighting}
\end{Shaded}

\begin{verbatim}
## Warning in par(mflow = c(1, 2)): "mflow" is not a graphical parameter
\end{verbatim}

\begin{Shaded}
\begin{Highlighting}[]
\FunctionTok{hist}\NormalTok{(x)}
\end{Highlighting}
\end{Shaded}

\includegraphics{class1_practice_files/figure-latex/unnamed-chunk-24-1.pdf}

\begin{Shaded}
\begin{Highlighting}[]
\FunctionTok{hist}\NormalTok{(x, }\AttributeTok{breaks=}\DecValTok{2}\NormalTok{)}
\end{Highlighting}
\end{Shaded}

\includegraphics{class1_practice_files/figure-latex/unnamed-chunk-24-2.pdf}

\begin{Shaded}
\begin{Highlighting}[]
\FunctionTok{hist}\NormalTok{(x, }\AttributeTok{breaks=}\DecValTok{100}\NormalTok{)}
\end{Highlighting}
\end{Shaded}

\includegraphics{class1_practice_files/figure-latex/unnamed-chunk-24-3.pdf}

\hypertarget{density-plot}{%
\section{Density plot}\label{density-plot}}

\begin{Shaded}
\begin{Highlighting}[]
\NormalTok{x}\OtherTok{=} \FunctionTok{runif}\NormalTok{(}\DecValTok{100}\NormalTok{,}\DecValTok{0}\NormalTok{,}\DecValTok{1}\NormalTok{)}

\FunctionTok{hist}\NormalTok{(x, }\AttributeTok{freq =} \ConstantTok{FALSE}\NormalTok{)}
\CommentTok{\# hist(x, breaks=2)}
\CommentTok{\# hist(x, breaks=100)}
\NormalTok{f}\OtherTok{=}\FunctionTok{density}\NormalTok{(x)}
\FunctionTok{lines}\NormalTok{(f,}\AttributeTok{col=}\StringTok{"red"}\NormalTok{,}\AttributeTok{lty=}\DecValTok{2}\NormalTok{,}\AttributeTok{lwd=}\FloatTok{2.5}\NormalTok{)}
\end{Highlighting}
\end{Shaded}

\includegraphics{class1_practice_files/figure-latex/unnamed-chunk-25-1.pdf}

\hypertarget{qq-plot}{%
\section{QQ PLOT}\label{qq-plot}}

\begin{Shaded}
\begin{Highlighting}[]
\NormalTok{x}\OtherTok{=} \FunctionTok{runif}\NormalTok{(}\DecValTok{100}\NormalTok{,}\DecValTok{0}\NormalTok{,}\DecValTok{1}\NormalTok{)}
\FunctionTok{qqnorm}\NormalTok{(x)}
\end{Highlighting}
\end{Shaded}

\includegraphics{class1_practice_files/figure-latex/unnamed-chunk-26-1.pdf}

\hypertarget{pie-charts}{%
\section{pie charts}\label{pie-charts}}

\begin{Shaded}
\begin{Highlighting}[]
\FunctionTok{pie}\NormalTok{(}\FunctionTok{summary}\NormalTok{(murders}\SpecialCharTok{$}\NormalTok{region))}
\end{Highlighting}
\end{Shaded}

\includegraphics{class1_practice_files/figure-latex/unnamed-chunk-27-1.pdf}

\hypertarget{pair-plot}{%
\section{Pair Plot}\label{pair-plot}}

\begin{Shaded}
\begin{Highlighting}[]
\NormalTok{x}\OtherTok{\textless{}{-}}\NormalTok{murders[,}\DecValTok{4}\SpecialCharTok{:}\DecValTok{6}\NormalTok{]}
\FunctionTok{pairs}\NormalTok{(x,}\AttributeTok{col=} \FunctionTok{as.numeric}\NormalTok{(murders}\SpecialCharTok{$}\NormalTok{region),}\AttributeTok{pch=}\FunctionTok{as.numeric}\NormalTok{(murders}\SpecialCharTok{$}\NormalTok{region))}
\end{Highlighting}
\end{Shaded}

\includegraphics{class1_practice_files/figure-latex/unnamed-chunk-28-1.pdf}

\end{document}
